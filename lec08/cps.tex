\chapter{Continuation-Passing Style and Defunctionalisation}\label{chapter:cps}

\dictum[Andrew Appel]{Continuation-passing style (CPS) is a program notation that makes every aspect of control flow and data flow explicit.}

\dictum[Timothy Griffin]{The cps transform is a logical embedding.}

\section{Overview}
Before we begin this chapter, let's take stock of what we have done, our goals for this chapter, and how we plan on achieving them. 

\subsection{Recap}
We have developed \textbf{\texttt{Interpreter 0}}, a ``\emph{denotational} interpreter'' for \texttt{Slang}. It is \emph{denotational} in the sense that it is ``definitional'', and so we can see at a glance that it is correct. For example, let's recall how we interpret \texttt{If}

\begin{code}[Interpreting \texttt{If} in \texttt{\textbf{Interpreter 0}}]
\begin{minted}[bgcolor=backcolour, linenos]{ocaml}
let rec interpret (e, env, store) =
  match e with
  | If(e1, e2, e3) -> 
    let (v, store') = interpret(e1, env, store) in
    match v with 
      | true -> interpret(e2, env, store')
      | false -> interpret(e3, env, store')
  | ...
\end{minted}
\end{code}

There's great simplicity in this -- so why do we have to do any more? Why can we not just stop here? The answer is that we are implicitly relying on many \texttt{OCaml} facilities, in particular, we're relying on the \texttt{OCaml} call stack. So really we've been cycling with the training wheels on, in some sense. It is time to take them off. 

\subsection{Goals}
The goal of this chapter, therefore, is to \emph{take the training wheels off}: liberate ourselves from the \texttt{OCaml} call stack and \emph{use an explicit call stack that we have full control over}. Doing this to \textbf{\texttt{Interpreter 0}} will create \textbf{\texttt{Interpreter 1}}.

\subsection{Roadmap}
Taking off the training wheels may seem like a really daunting task. In this chapter, what we're going to do is gradually build up to a  a general strategy for doing so, by considering concrete, simple examples.

First, in \Cref{section:call-stacks}, we're going to examine \emph{what a call stack is}, in order to understand what it is that \texttt{OCaml} has given us for free, and what it is we'll need to construct if we want to build it on our own.

Second, in \Cref{section:cps-transform}, we'll liberate ourselves from relying on the call stack. The main idea here is to convert a \emph{recursive} function to an \emph{tail-recursive} one. The \emph{continuation-passing style} transformation will give us such a technique.

Third, we'll see that CPS alone will end up giving us something that has the structure of a call stack, but is not \emph{quite} one. In \Cref{section:defunctionalisation}, we'll turn it into something more stack like using \emph{defunctionalisation}. 

Finall, in \Cref{section:clean-up}, we'll perform some clean-up, eliminating \emph{mutual recursion} and then doing some refactoring. 



\section{Call Stacks}\label{section:call-stacks}
Given our goal: to switch from \texttt{OCaml}'s call stack to our own, explicit call stack, it's first important to \emph{truly} understand what a stack is. To do this, let's examine a simple function that sums the elements of a list, and let's sketch how the stack changes when we run the function on the input list \texttt{[1; 2; 3]}.

\begin{code}[A function to sum the elements of a list in \texttt{OCaml}]
\label{code:sum-ocaml}
\begin{minted}[bgcolor=backcolour, linenos]{ocaml}
let rec sum xs = 
   match xs with
     | [] -> 0
     | x::xs -> x + sum xs
\end{minted}
\end{code}

In the first call, we start with an empty stack and call \texttt{sum [1;2;3]}. \texttt{sum [1;2;3]} will evaluate to \texttt{1 + sum [2;3]}. Let's break this into two parts. We are \emph{first} going to evaluate \texttt{sum [2;3]}, and then \emph{second}, take the result of that evaluation and add \texttt{1} to it. Before we dive in headfirst in evaluating \texttt{sum [2;3]}, we need to use some scratch space to note down what we need to do once we're done: specifically, we need to add \texttt{1}. 

That's exactly the purpose of the call stack --- we push an \emph{activation frame} onto the call stack, in this case, I hide away the specific details of the activation frame and focus on what it \emph{is} --- a note that we should \texttt{ADD 1} on completion of the recursive call. It's important to get the intuition for what an activation frame is: an activation frame represents \emph{the rest of the computation}. Pushing an activation frame onto a call stack is thus just nerd-speak for ``jotting down what we need to do so we don't forget it''. 

\begin{figure}[h]
    \centering
    \resizebox{\textwidth}{!}{%
\begin{tikzpicture}
\definecolor{mdgrey}{HTML}{9696A0}
\definecolor{ltgrey}{HTML}{D5D5DC}
\begin{scope}[execute at end node=\strut]
    \node[text width=30mm, align = center] (call1) at (0, 0) {\texttt{sum [1;2;3]}};
    
    \node [above = 0.2 of call1, rectangle, text width = 15mm, fill = ltgrey, draw=ltgrey, line width=0.3mm, minimum height = 40mm, align = center] (stack1) {\textcolor{mdgrey}{\sffamily{\textbf{Stack Space}}}};


    \node[right = 6mm of call1, text width=30mm, align = center] (call2){\texttt{sum [2;3]}};
    
    \node [above = 0.2 of call2, rectangle, text width = 15mm, fill = ltgrey, draw=ltgrey, line width=0.3mm, minimum height = 40mm, align = center] (stack2) {\textcolor{mdgrey}{\sffamily{\textbf{Stack Space}}}};

    \node [above = 0.2 of call2, rectangle, text width = 15mm, fill = white, draw, line width=0.3mm, minimum height = 8mm, align = center] (stack21) {\texttt{ADD 1}};


    \node[right = 6mm of call2, text width=30mm, align = center] (call3){\texttt{sum [3]}};
    
    \node [above = 0.2 of call3, rectangle, text width = 15mm, fill = ltgrey, draw=ltgrey, line width=0.3mm, minimum height = 40mm, align = center] (stack3) {\textcolor{mdgrey}{\sffamily{\textbf{Stack Space}}}};

    \node [above = 0.2 of call3, rectangle, text width = 15mm, fill = white, draw, line width=0.3mm, minimum height = 8mm, align = center] (stack31) {\texttt{ADD 1}};

    \node [above = -0.3mm of stack31, rectangle, text width = 15mm, fill = white, draw, line width=0.3mm, minimum height = 8mm, align = center] (stack32) {\texttt{ADD 2}};

     \node[right = 6mm of call3, text width=30mm, align = center] (call4){\texttt{sum []}};
    
    \node [above = 0.2 of call4, rectangle, text width = 15mm, fill = ltgrey, draw=ltgrey, line width=0.3mm, minimum height = 40mm, align = center] (stack4) {\textcolor{mdgrey}{\sffamily{\textbf{Stack Space}}}};

    \node [above = 0.2 of call4, rectangle, text width = 15mm, fill = white, draw, line width=0.3mm, minimum height = 8mm, align = center] (stack41) {\texttt{ADD 1}};

    \node [above = -0.3mm of stack41, rectangle, text width = 15mm, fill = white, draw, line width=0.3mm, minimum height = 8mm, align = center] (stack42) {\texttt{ADD 2}};

    \node [above = -0.3mm of stack42, rectangle, text width = 15mm, fill = white, draw, line width=0.3mm, minimum height = 8mm, align = center] (stack43) {\texttt{ADD 3}};

    \node[right = 6mm of call4, text width=30mm, align = center] (app0) {\texttt{APPLY}};
    
    \node [above = 0.2 of app0, rectangle, text width = 15mm, fill = ltgrey, draw=ltgrey, line width=0.3mm, minimum height = 40mm, align = center] (stack5) {\textcolor{mdgrey}{\sffamily{\textbf{Stack Space}}}};

    \node [above = 0.2 of app0, rectangle, text width = 15mm, fill = white, draw, line width=0.3mm, minimum height = 8mm, align = center] (stack51) {\texttt{ADD 1}};

    \node [above = -0.3mm of stack51, rectangle, text width = 15mm, fill = white, draw, line width=0.3mm, minimum height = 8mm, align = center] (stack52) {\texttt{ADD 2}};

    \node [above = -0.3mm of stack52, rectangle, text width = 15mm, fill = white, draw, line width=0.3mm, minimum height = 8mm, align = center] (stack53) {\texttt{ADD 3}};

    \node [above = -0.3mm of stack53, rectangle, text width = 15mm, fill = white, draw, line width=0.3mm, minimum height = 8mm, align = center] (stack54) {\texttt{VAL 0}};
    
    \node[right = 6mm of app0, text width=30mm, align = center] (eval1) {\texttt{EVAL}};
    
    \node [above = 0.2 of eval1, rectangle, text width = 15mm, fill = ltgrey, draw=ltgrey, line width=0.3mm, minimum height = 40mm, align = center] (stack6) {\textcolor{mdgrey}{\sffamily{\textbf{Stack Space}}}};

    \node [above = 0.2 of eval1, rectangle, text width = 15mm, fill = ltgrey, draw=ltgrey, line width=0.3mm, minimum height = 40mm, align = center] (stack5) {\textcolor{mdgrey}{\sffamily{\textbf{Stack Space}}}};

    \node [above = 0.2 of eval1, rectangle, text width = 15mm, fill = white, draw, line width=0.3mm, minimum height = 8mm, align = center] (stack61) {\texttt{ADD 1}};

    \node [above = -0.3mm of stack61, rectangle, text width = 15mm, fill = white, draw, line width=0.3mm, minimum height = 8mm, align = center] (stack62) {\texttt{ADD 2}};

    \node [above = -0.3mm of stack62, rectangle, text width = 15mm, fill = white, draw, line width=0.3mm, minimum height = 8mm, align = center] (stack63) {\texttt{VAL 3}};

    \node[right = 6mm of eval1, text width=30mm, align = center] (eval2){\texttt{EVAL}};
    
    \node [above = 0.2 of eval2, rectangle, text width = 15mm, fill = ltgrey, draw=ltgrey, line width=0.3mm, minimum height = 40mm, align = center] (stack7) {\textcolor{mdgrey}{\sffamily{\textbf{Stack Space}}}};

    \node [above = 0.2 of eval2, rectangle, text width = 15mm, fill = white, draw, line width=0.3mm, minimum height = 8mm, align = center] (stack71) {\texttt{ADD 1}};

    \node [above = -0.3mm of stack62, rectangle, text width = 15mm, fill = white, draw, line width=0.3mm, minimum height = 8mm, align = center] (stack72) {\texttt{VAL 5}};

    \node[right = 6mm of eval2, text width=30mm, align = center] (eval3){\texttt{EVAL}};
    
    \node [above = 0.2 of eval3, rectangle, text width = 15mm, fill = ltgrey, draw=ltgrey, line width=0.3mm, minimum height = 40mm, align = center] (stack8) {\textcolor{mdgrey}{\sffamily{\textbf{Stack Space}}}};

    \node [above = 0.2 of eval3, rectangle, text width = 15mm, fill = white, draw, line width=0.3mm, minimum height = 8mm, align = center] (stack81) {\texttt{VAL 6}};
    
\end{scope}
\end{tikzpicture}
}
    \vspace{-10mm}
    \caption{Stack evolution when evaluating 
    \texttt{sum [1;2;3]}}
    \label{fig-sumstack}
\end{figure}

As we continue making calls, we'll push \texttt{ADD 2} and \texttt{ADD 3} onto the stack, at which point we hit the base case. Things start to change when we reach the base case -- at this point, we've fully \emph{unravelled} the computation into a sequence of activation frames. We know exactly what we need to do once we have a value, but we don't yet have a value. But luckily, we've hit the base case! So we can simply \emph{pass in} the value (in this case, \texttt{0}), using the \emph{remembered transformations} on the stack to transform the value. After this point, the stack only shrinks as we apply transformation after transformation.

This is summarised graphically in \Cref{fig-sumstack}. It's an extremely important point: when we \emph{build} the stack, we are \emph{remembering what we need to do once we hit a base case}. Don't be fooled by the scary terminology. Once we hit the base case, we simply \emph{unwind} the stack by using the remembered operations to transform our value.

\section{Continuation Passing Style}\label{section:cps-transform}
Our first step is to stop using the call stack, by converting recursive functions into tail recursive ones. Continuation Passing Style (CPS) provides us with a technique to convert any \emph{recursive} function into a \emph{tail-recursive} one. In \Cref{subsection:tailrecursion}, we'll first quickly recap tail recursion. In \Cref{subsection:cps-transform,subsection:cps-fib}, we'll use the \texttt{sum} and \texttt{fibonacci} functions to illustrate how to perform the CPS transform, and provide intuition as to what it does. Finally, in \Cref{subsection:cps-benefits} we'll look at benefits of the CPS transform beyond our very narrow setting.

\subsection{Tail Recursion}\label{subsection:tailrecursion}
We'll quickly recap tail recursion, which was covered in {\sffamily Part IA Foundations of Computer Science}. To tell if a recursive function is tail recursive, just ask yourself: “Whenever I make a recursive call, do I need to do anything after the call returns?”

If the answer is yes, then the function is \textbf{not} tail recursive. Hence, the \texttt{sum} function defined in \Cref{code:sum-ocaml} is not tail recursive, because the recursive expression is \texttt{x + sum xs}. Once the recursive call \texttt{sum xs} returns, there is still work to do in adding \texttt{x}.

If the answer is no, then the function \emph{is} tail recursive. An easy example of a tail recursive function is checking if an element is in a list

\begin{code}[A tail recursive program for checking if an element is in a list in \texttt{OCaml}]
\label{code:sum-cps-ocaml}
\begin{minted}[bgcolor=backcolour, linenos]{ocaml}
let rec mem x xs = 
  match xs with
    | [] -> False
    | y::xs -> x == y || mem x xs 
\end{minted}
\end{code}

This is tail recursive because either \texttt{x == y}, in which case we can return true without having to perform the recursive call, or \texttt{x != y}, in which case we can perform the recursive call and simply return the result. 

\subsection{CPS Transformation}\label{subsection:cps-transform}
Let's consider, once again the \texttt{sum} function defined in \Cref{code:sum-ocaml}. This is \emph{not} tail recursive, since once the recursive call terminates, we still have work to do! Specifically, we need to \texttt{ADD x} once the recursive call is done. 

Let's describe this annoying ``extra work'' in a way that makes its nature more apparent. Rather than saying \texttt{x + sum xs}, we'll say \texttt{(fun res -> x + res) (sum xs)} (where \texttt{res} stands for result). Putting it together, 

\begin{code}[Making the order of evaluation explicit]
\label{code:sum-ocaml}
\begin{minted}[bgcolor=backcolour, linenos]{ocaml}
let rec sum xs = 
   match xs with
     | [] -> 0
     | x::xs -> (fun res -> x + res) (sum xs)
\end{minted}
\end{code}

If you remember call-by-value semantics, this makes the order of evaluation explicit. We \emph{first} evaluate \texttt{sum xs} and \emph{then} apply the function \texttt{(fun res -> x + res) (sum xs)} to it. 

The key insight behind the CPS transform is that \textbf{functions are values}. Hence, they can be passed to other functions as arguments. Therefore, in order to make the function tail recursive, we can \emph{pass in the work to do to the recursive call}. That is, we'll rewrite \texttt{sum} to take in an additional argument, called \texttt{k}, that represents ``the work that's left to be done''. At every iteration, we'll update ``the work that's left to be done'', until we hit the base case, at which point we'll simply execute it. This is exactly the CPS transform, and the CPS transformed \texttt{sum} function can be seen in \Cref{code:sum-cps-ocaml}.

\begin{code}[A CPS-transformed function to sum the elements of a list in \texttt{OCaml}]
\label{code:sum-cps-ocaml}
\begin{minted}[bgcolor=backcolour, linenos]{ocaml}
let rec sum_cps xs k = 
   match xs with
     | [] -> k 0
     | x::xs -> sum_cps xs (fun res -> k (x + res))
\end{minted}
\end{code}

As we can see on \texttt{Line 4}, this is now tail recursive --- we've taken the additional work to be done, bundled it up into a function, and passed it as a second argument. 

There is a subtlety --- why is it \texttt{fun res -> k (x + res)} and not \texttt{fun res -> x + res}? Think about what each function call receives --- a \texttt{k} from its caller, that represents what its caller needs to do. Think about what each function call gives its callee --- some \texttt{k} that \emph{chains} what the function call needs to do with what the function caller needed to do. In doing so, we are building up ``the rest of the computation''. Here, we chain by using function composition -- add \texttt{res} to \texttt{x}, and then execute whatever our function caller wants us to do.

Further, when we reach a value, as in the base case, we now \emph{apply the fully built-up computation} to the value. That's what we're doing in \texttt{Line 3}.

In fact, we can give the ``rest of the computation'' a name -- it is the \textbf{continuation}. Since we are \emph{passing} the continuation along at each step, this is called \textbf{Continuation Passing Style}.

Let's visualise how the continuation changes when we apply \texttt{sum\_cps} to an initial argument \texttt{[1;2;3]} and initial continuation \texttt{fun x -> x}. For space reasons, we'll use a slightly different naming convention for functions. We'll take a function \texttt{fun x -> fun y -> x + y} and write it as such $\lambda x. \lambda y. x + y$. 

We'll consider two stages: in the first, we'll \texttt{CALL} \texttt{sum\_cps} recursively, building up the continuation.

\begin{tabular}{ll}
\textbf{Arguments} & \textbf{Continuation}\\ \hline
\texttt{[1;2;3]} & $\lambda x. x$ \\
\texttt{[2;3]} & $\lambda r_1. (\lambda x. x) (r_1 + 1)$\\
\texttt{[3]} & $\lambda r_2. (\lambda r_1. (\lambda x. x) (r_1 + 1)) (r_2 + 2)$\\
\texttt{[]} & $\lambda r_3. (\lambda r_2. (\lambda r_1. (\lambda x. x) (r_1 + 1)) (r_2 + 2)) (r_3 + 3)$\\
\end{tabular}

In the second, we'll \texttt{EVAL}-uate the fully built-up continuation until we get a singular value. 

\begin{tabular}{ll}
\textbf{Continuation} & \textbf{Value}\\ \hline
$\lambda r_3. (\lambda r_2. (\lambda r_1. (\lambda x. x) (r_1 + 1)) (r_2 + 2)) (r_3 + 3)$ & $0$ \\
$\lambda r_2. (\lambda r_1. (\lambda x. x) (r_1 + 1)) (r_2 + 2)$ & $3$\\
$\lambda r_1. (\lambda x. x) (r_1 + 1)$ & $5$ \\
$\lambda x. x$ & $6$
\end{tabular}

Notice how we use the identity function \texttt{fun x -> x} to signify that \emph{nothing further needs to be done}. Using this, we can bundle \texttt{sum\_cps} into a single argument function that users don't need to explicitly pass a continuation to. 

\begin{code}[Wrapping the \texttt{sum\_cps} function]
\label{code:sum-cps-wrapped-ocaml}
\begin{minted}[bgcolor=backcolour, linenos]{ocaml}
let sum xs = 
  let rec sum_cps xs k = 
    match xs with
      | [] -> k 0
      | x::xs -> sum_cps xs (fun res -> k (x + res))
    in
    sum_cps xs (fun x -> x)
\end{minted}
\end{code}

We'd like you to note two things: first, that cps has converted a non-tail recursive function into a tail-recursive one, as promised. Second, the continuation \emph{exactly mirrors} the stack! Building up the continuation, using function composition, corresponds to pushing \emph{activation frames} onto the stack. Applying the continuation to a value corresponds to the \texttt{APPLY} step. The value is then successively transformed by the functions in the continuation. This is not a coincidence -- you can, and \emph{should}, think of the continuation as a \emph{functionalised stack}. 

\subsection{CPS transforming the fibonacci function}
\label{subsection:cps-fib}
\begin{code}[A fibonacci function in \texttt{OCaml}]
\label{code:fib-ocaml-again}
\begin{minted}[bgcolor=backcolour, linenos]{ocaml}
let rec fib n = match n with
  | 0 -> 1
  | 1 -> 1
  | n -> fib n-1 + fib n-2
\end{minted}
\end{code}

\begin{code}[A CPS-transformed fibonacci function in \texttt{OCaml}]
\label{code:fib-ocaml-again}
\begin{minted}[bgcolor=backcolour, linenos]{ocaml}
let rec fib_cps n k = match n with
  | 0 -> k 1
  | 1 -> k 1
  | n -> fib_cps n-1 (fun x -> fib_cps n-2 (fun y -> k (x + y)))
\end{minted}
\end{code}

The CPS transformation here is slightly more involved. Two points: first, we have two base cases rather than one. Second, the CPS transformed recursive case looks rather gnarly. But it isn't too bad! Let's start from first principles. We wish to make the following expression tail recursive 

\begin{minted}[bgcolor=backcolour]{ocaml}
fib n-1 + fib n-2
\end{minted}
To do that, let's bundle up what needs to be done into a function, just like we did for \texttt{sum}
\begin{minted}[bgcolor=backcolour]{ocaml}
(fun res -> res + fib n-2) fib n-1 
\end{minted}
But the body of the resultant function features another recursive call, so we repeat this one more time
\begin{minted}[bgcolor=backcolour]{ocaml}
(fun res1 -> (fun res2 -> res1 + res2) fib n-2) fib n-1 
\end{minted}
If we replace \texttt{res1} with \texttt{x} and \texttt{res2} with \texttt{y}, we get
\begin{minted}[bgcolor=backcolour]{ocaml}
(fun x -> (fun y -> x + y) fib n-2) fib n-1 
\end{minted}
The rest of the conversion is simply reorganisation and sticking the continuation of our caller in at the right place.
\begin{minted}[bgcolor=backcolour]{ocaml}
fib n-1 (fun x -> fib n-2 (fun y -> k (x + y))) 
\end{minted}


\subsection{Benefits of Continuations}\label{subsection:cps-benefits}
Continuations are \emph{incredibly powerful}. Let's list their merits, and then dig through them in greater detail. Lots of the finer technical details here will be non-examinable, but the key ideas are!

\subsubsection{Every call is a tail call}
We've seen this in the previous section.

\subsubsection{Every intermediate result is named}
In \Cref{code:sum-cps-ocaml}, we've chosen \texttt{res} as the name of the intermediate result of \texttt{sum xs}.

{\sffamily \textbf{(Non-Examinable)} CPS and SSA.} In addition, it is not \emph{just} that every intermediate result is named, but every function in the continuation is applied \emph{at most once}. In {\sffamily Part II Optimising Compilers}, you'll learn about \textit{Static Single Assignment} (SSA) form, a transformation commonly used in imperative intermediate representations. SSA form ensures that every user variable is assigned \textit{at most once}. For example, the following code is \textbf{not} in SSA form, since (for example) \texttt{a} is assigned to twice -- once the result of \texttt{f(1)}, and once the result of \texttt{f(6)}.

\begin{code}[\texttt{C} Code that is not in SSA form]
\label{code:fib-ocaml-again}
\begin{minted}[bgcolor=backcolour, linenos]{C}
extern int f(int); 
extern void h(int,int); 
void g() {   
    int a,b,c;   
    a = f(1); b = f(2); h(a,b);   
    b = f(3); c = f(4); h(b,c);   
    c = f(5); a = f(6); h(c,a); 
}
\end{minted}
\end{code}

To transform this code into SSA form, we number every assignment, so every variable is assigned to \emph{at most} once. 

\begin{code}[\texttt{C} Code that is in SSA form]
\label{code:fib-ocaml-again}
\begin{minted}[bgcolor=backcolour, linenos]{C}
extern int f(int); 
extern void h(int,int); 
void g() {   
    int a1,a2,b1,b2,c1,c2;   
    a1 = f(1); b1 = f(2); h(a1,b1);   
    b2 = f(3); c1 = f(4); h(b2,c1);   
    c2 = f(5); a2 = f(6); h(c2,a2); 
}
\end{minted}
\end{code}

In CPS, every function is applied at most once, so every function variable is bound at most once. In SSA, every variable is assigned at most once. CPS and SSA might seem related, and in fact, they are. \citet{kelsey-1995} shows a correspondence between CPS and SSA. 

\subsubsection{The evaluation order is explicit}
This is apparent when we consider the fibonacci function. We start off with an expression
\begin{minted}[bgcolor=backcolour]{ocaml}
fib n-1 + fib n-2
\end{minted}
Where it is not at all clear what the order of evaluation is, just from the syntax. We end up with the expression,
\begin{minted}[bgcolor=backcolour]{ocaml}
fib n-1 (fun x -> fib n-2 (fun y -> k (x + y))) 
\end{minted}
Now, knowing nothing about \texttt{OCaml}, it is apparent that we first evalute \texttt{fib n-1}, then \texttt{fib n-2}, and then add their results, and then apply whatever's left of the computation. 

{\sffamily \textbf{(Non-Examinable)}: Fused Lexing and Parsing.} It turns out that this simple idea can be used in very powerful ways! Let's revisit, for a brief moment, lexing and parsing. When we talked about the lexer and the parser, we described them as communicating through an interface --- the lexer takes a character stream and turns it into a token stream, which is then handled by a parser. Interfaces are great for separating concerns --- we don't want to worry too much about the parser when defining the lexer, and we don't want to worry too much about the lexer when defining the parser. But interfaces also block off opportunities for optimisation --- if the lexer and the parser were defined together, then whenever the lexer emits a token, the parser could take it immediately, and put it \emph{exactly} where it needs to be in the AST, and then return control back to the lexer, in a way that \textit{exposes more opportunities for optimisation}. It'd be really cool if there was a way to define lexers and parsers separately, and then \emph{automatically} fuse them. \citet{yallop-2023} illustrate that this is possible, and one of the key ideas is to CPS-transform the lexer, and then CPS-transform the parser, and then \emph{interleave} their continuations. This is made possible because the order of evaluation has been made explicit. 

\subsubsection{Every continuation is reified}
What this means is, in essence, the program's \emph{call stack} has become a real object that can be manipulated by the program. 

{\sffamily \textbf{(Non-Examinable)} Fast multiplication and Backtracking}. Let's consider how we might, and why we might want to, manipulate continuations. Here are two examples

First, fast multiplication of a list. Here's a standard way of multiplying the elements of a list 

\begin{code}[Multiplying the elements of a list in \texttt{OCaml}]
\label{code:mul-ocaml}
\begin{minted}[bgcolor=backcolour, linenos]{ocaml}
let rec mul xs = match xs with
  | [] -> 1
  | x::xs -> x * mul xs
\end{minted}
\end{code}

Which we CPS-convert

\begin{code}[A CPS-transformed function in \texttt{OCaml} that multiplies the elements of a list]
\label{code:mul-ocaml-fast}
\begin{minted}[bgcolor=backcolour, linenos]{ocaml}
let rec mul_cps xs k = match xs with
  | [] -> k 1
  | x::xs -> mul_cps xs (fun res -> k (x * res))
\end{minted}
\end{code}

We can make this multiplication faster, by simply returning zero whenever we encounter a zero. 

\begin{code}[A faster \texttt{OCaml} function that multiplies the elements of a list]
\label{code:mul-ocaml-fastest}
\begin{minted}[bgcolor=backcolour, linenos]{ocaml}
let rec mul_cps xs k = 
  match xs with
    | [] -> k 1
    | 0::xs -> k 0
    | x::xs -> mul_cps xs (fun res -> k (x * res))
\end{minted}
\end{code}

But this is still not the most optimal, since we will have to go through the \emph{built-up} continuation \texttt{k}, which may be some non-zero sequence of multiplications, before returning. 

However, since continuations are now explicit objects that we can manipulate, let's \emph{save} the continuation \emph{before} we do any multiplication, and call it \texttt{u}. If we reach a \texttt{0}, let's not apply the built up continuation \texttt{k}, but instead the saved continuation \texttt{u}.

\begin{code}[An even faster \texttt{OCaml} function that multiplies the elements of a list]
\label{code:fib-ocaml-again}
\begin{minted}[bgcolor=backcolour, linenos]{ocaml}
let mul_cps xs u = 
  let rec mul_cps_inner xs k = 
    match xs with
      | [] -> k 1
      | 0::xs -> u 0
      | x::xs -> mul_cps_inner xs (fun res -> k (x * res))
  in
  mul_cps_inner xs u
\end{minted}
\end{code}

Second, we can implement \texttt{Prolog} style backtracking in \texttt{OCaml}. Specifically, we implement an \texttt{amb} primitive, described first by \citet{mccarthy-1963} in 1961, that takes in a list of values it could be, and tries them in order. For example, in \Cref{code:amb-specification}, \texttt{x} first takes the value \texttt{1}, then \texttt{2}, then \texttt{3}, etcetera.

\begin{code}[An example of the \texttt{amb} primitive in operation]
\label{code:amb-specification}
\begin{minted}[bgcolor=backcolour, linenos]{ocaml}
let x = amb [1;2;3;4] in
let y = amb [5;6;7;8] in
let _ = assert x+y == 11 in
(x, y) (* returns (3, 8) *)
\end{minted}
\end{code}

We can implement the \texttt{amb} primitive using continuations. We won't go through the details here (though we will attach the code for your perusal). You'll learn much more about \texttt{amb} in {\sffamily Part II Types}, but the key is that every time we encounter a decision point that we may need to re-consider during a backtracking phase, we can push the \emph{continuation} (representing the call stack) onto a \emph{stack}. If we fail, we can simply pop the last decision point off the stack. 

\begin{code}[Implementing the \texttt{amb} primitive]
\label{code:amb-specification}
\begin{minted}[bgcolor=backcolour, linenos]{ocaml}
module AMB : sig
  (* Interface *)
  exception AmbFail
  val assert : bool -> unit
  val amb : int list -> int
end = struct
  (* Implementation *)
  exception AmbFail
  let stack: int option cont list ref = ref []
  let fail () = match !stack with
    | [] -> raise AmbFail
    | k::ks -> let _ = stack := ks in (*Remove the last choice point*)
               throw k NONE (*Resume executing from the choice point*)
  let assert b = if b then () else fail ()
  let amb xs = match xs with
    | [] -> raise AmbFail
    | x::xs -> let next k = 
                 let _ = stack := k::(!stack) in 
                 (*Push the current continuation onto the stack*)
                 SOME x
               in
               match callcc (fun k -> next k) with
                 | SOME v -> v 
                 | NONE -> amb xs 
end
\end{minted}
\end{code}

\subsection{The CPS transform as a logical embedding$^{*}$}
In \texttt{Haskell}, the way you work with continuations is by using two primitives, \texttt{callcc} (``call with current continuation'') and \texttt{throw}. \texttt{callcc} allows the programmer to \emph{name} the current continuation (call stack) and allows your code to reference that continuation. \texttt{throw} is a way of passing a value to a continuation. Fast multiplication with \texttt{callcc} and \texttt{throw} looks like

\begin{code}[Fast multiplication with \texttt{callcc} and \texttt{throw}]
\label{code:mul-ocaml-callcc}
\begin{minted}[bgcolor=backcolour, linenos]{ocaml}
let mul_callcc xs = 
  let rec mul_inner xs k = 
    match xs with
      | [] -> 1
      | 0::xs -> throw u 0
      | x::xs -> x * mul_inner xs u
  in callcc (fun u -> mul_inner xs u)
\end{minted}
\end{code}

For simplicity, \texttt{mul\_inner} is not CPS converted. 

In the 1970s, there was a widespread belief that \texttt{callcc} ``couldn't be typed''. \citet{griffin-1990} was the first to discover that it \emph{could} be typed. If I let the return type be $A$, and the type of a continuation that expects an element of $A$ be $A \rightarrow B$, then the type of \texttt{callcc} is
\[((A \rightarrow B) \rightarrow A)) \rightarrow A\]
Basically, if
\begin{enumerate}
    \item \texttt{u} is a continuation expecting an element of $A$ (\texttt{u} has type $A \rightarrow B)$,
    \item Assuming \texttt{u} has type $A \rightarrow B$, can conclude that \texttt{mul\_callcc xs u} has type $A$,
    \item Then, \texttt{callcc (fun u -> mul\_callcc xs u)} has type $A$
\end{enumerate} 

Eagle-eyed readers will spot that this corresponds to Pierce's Law, 
\[((P \rightarrow Q) \rightarrow P)) \rightarrow P\]
Pierce's Law is an axiom of \emph{Classical Logic} but not of \emph{Constructive Logic}. Don't worry if you don't know what this is, you'll learn about it in {\sffamily Part II Types}. However, if you are curious, Classical Logic is a logic of \emph{fact}, whereas Constructive Logic is a logic of \emph{data}. In order to \emph{prove} something in Constructive Logic, you need to give a \emph{decision algorithm}. This is not needed in Classical Logic. Consider the statement 

\begin{center}
    Every program either terminates or does not terminate.
\end{center}

In Classical Logic, we know factually, that this is true (and it is provable). In Constructive Logic, in order to prove this to be true, we need to produce an \emph{algorithm} that takes a program and tells us, for any program, if it terminates or does not terminate (you can prove such an algorithm does not exist, see {\sffamily Part IB Computation Theory}). It turns out that the only axiom you need to add to Constructive Logic to turn it into Classical Logic is Pierce's Law. 

The discovery that \texttt{callcc}'s type corresponded to Pierce's Law uncovered the \emph{logical} nature of CPS. In {\sffamily Part IB Semantics of Programming Languages}, you'll have touched on the \emph{Curry-Howard Correspondence} --- every concept in programming has a corresponding concept in logic. It turns out that the CPS transform corresponds to a way to \emph{logically embed} Classical Logic into Constructive Logic.

All of this is non-examinable, but if you're interested, you can find out much more in {\sffamily Part II Types}.

\section{Defunctionalisation}\label{section:defunctionalisation}
We've seen that the CPS transformation produces continuations, that logically, \emph{are} call stacks. But this is a course on compilers, and it is not enough to logically \emph{be} the call stack. The biggest \emph{structural} difference now is that the continuation is a deeply nested, high-order function, and stacks are flat lists of commands. 

\emph{Defunctionalisation} is a technique for getting rid of higher order functions, and thus is exactly what we need in this scenario. We'll first illustrate the technique on simple \texttt{filter} function. Following this, in \Cref{subsection:defun-with-cps}, we'll show how to use defunctionalisation after CPS to produce a call stack with the right structure. Finally, in \Cref{subsection:defun-benefits}, we'll consider the benefits of defunctionalisation beyond our narrow setting.

\subsection{Introducing Defunctionalisation}
Let's consider the following \texttt{OCaml} function that filters a list.

\begin{code}[An \texttt{OCaml} function for filtering a list]
\label{code:defun-0}
\begin{minted}[bgcolor=backcolour, linenos]{ocaml}
let rec filter p xs = match xs with
  | [] -> []
  | x::xs -> if p x then x::(filter xs) else filter xs

let f l y = filter (fun x -> x < 3) l @ filter (fun x -> x > y)
\end{minted}
\end{code}

We want to get rid of higher order functions, so in particular \texttt{fun x -> x < 3}. Here's one idea: let's introduce a set of constructors, one that represents each function. For example, we might introduce the constructor \texttt{Add\_one} to represent \texttt{fun x -> x + 1}, and \texttt{Sb of int} to represent \texttt{fun x -> x - y} (where \texttt{y} is in scope).

\begin{code}[Defunctionalisation: Step 1: Introducing constructors]
\label{code:defun-1}
\begin{minted}[bgcolor=backcolour, linenos]{ocaml}
type fn = Lt_three | Gt of int (* CHANGED *)

let rec filter p xs = match xs with
  | [] -> []
  | x::xs -> if p x then x::(filter xs) else filter xs

let f l y = filter (fun x -> x < 3) l @ filter (fun x -> x > y)
\end{minted}
\end{code}

Once we've added the type constructors, we've lost information! We know that when we see \texttt{Lt\_three}, we should apply the function \texttt{fun x -> x < 3}, but this isn't recorded anywhere in the code! Let's fix it, by defining an explicit \texttt{apply} function

\begin{code}[Defunctionalisation Step 2: Defining the \texttt{apply} function]
\label{code:defun-2}
\begin{minted}[bgcolor=backcolour, linenos]{ocaml}
type fn = Lt_three | Gt of int 

let apply fn x = match fn, x with (* CHANGED *)
  | Lt_three, x -> x < 3
  | Gt y, x     -> x > y
  
let rec filter p xs = match xs with 
  | [] -> []
  | x::xs -> if p x then x::(filter xs) else filter xs

let f l y = filter (fun x -> x < 3) l @ filter (fun x -> x > y)
\end{minted}
\end{code}

Finally, let's get rid of higher-order functions by replacing each function with its constructor, and replacing each \emph{call} of such a function with an explicit call to our \texttt{apply} function. 

\begin{code}[Defunctionalisation: Step 3]
\label{code:defun-3}
\begin{minted}[bgcolor=backcolour, linenos]{ocaml}
type fn = Lt_three | Gt of int 

let apply fn x = match fn, x with
  | Lt_three, x -> x < 3
  | Gt y, x     -> x > y
  
let rec filter p xs = match xs with
  | [] -> []
  | x::xs -> if apply p x then x::(filter xs) else filter xs (* CHANGED *)

let f l y = filter Lt_three l @ filter (Gt y) l (* CHANGED *)
\end{minted}
\end{code}

\subsection{Defunctionalising after CPS transform}\label{subsection:defun-with-cps}
Now that we've covered defunctionalisation as a technique, let's see how we can perform defunctionalisation on a CPS converted function, and how it helps us make the stack explicit. Specifically, we'll use the CPS transformed \texttt{fibonacci} function.

\begin{code}[A CPS-transformed fibonacci function in \texttt{OCaml}]
\label{code:defun-fib-0}
\begin{minted}[bgcolor=backcolour, linenos]{ocaml}
let rec fib_cps n k = match n with
  | 0 -> k 1
  | 1 -> k 1
  | n -> fib_cps n-1 (fun x -> fib_cps n-2 (fun y -> k (x + y)))

let fib_1 n = fib_cps n (fun x -> x)
\end{minted}
\end{code}

Let's apply the three steps. First, let's introduce three constructors, call them \texttt{ID}, \texttt{K1}, and \texttt{K2}, one for each of the anonymous functions. 

\begin{code}[Defunctionalisation Step 1]
\label{code:defun-fib-1}
\begin{minted}[bgcolor=backcolour, linenos]{ocaml}
type cont = ID | K1 of int * cont | K2 of int * cont

let rec fib_cps n k = 
  match n with
    | 0 -> k 1
    | 1 -> k 1
    | n -> fib_cps n-1 (fun x -> fib_cps n-2 (fun y -> k (x + y)))

let fib_1 n = fib_cps n (fun x -> x)
\end{minted}
\end{code}

Next, define the \texttt{apply\_cont} function. 

\begin{code}[Defunctionalisation Step 2]
\label{code:defun-fib-1}
\begin{minted}[bgcolor=backcolour, linenos]{ocaml}
type cont = ID | K1 of int * cont | K2 of int * cont

let rec apply_cont k v = match k, v with
  | ID, v -> v
  | K1(n, k), x -> fib_cps_defun (n-2) (K2(x, k))
  | K2(x, k), y -> apply_cont k (x+y)

and fib_cps n k = 
  match n with
    | 0 -> k 1
    | 1 -> k 1
    | n -> fib_cps n-1 (fun x -> fib_cps n-2 (fun y -> k (x + y)))

let fib_1 n = fib_cps n (fun x -> x)
\end{minted}
\end{code}

Finally, replace each function with its constructor. 

\begin{code}[Defunctionalisation Step 3]
\label{code:defun-fib-1}
\begin{minted}[bgcolor=backcolour, linenos]{ocaml}
type cont = ID | K1 of int * cont | K2 of int * cont

let rec apply_cont k v = match k, v with
  | ID, v -> v
  | K1(n, k), x -> fib_cps_defun (n-2) (K2(x, k))
  | K2(x, k), y -> apply_cont k (x+y)

and fib_cps_defun n k = 
  match n with
    | 0 -> apply_cont k 1
    | 1 -> apply_cont k 1
    | n -> fib_cps_defun n-1 (K1(n, k))

let fib_2 n = fib_cps_defun n ID
\end{minted}
\end{code}

\subsubsection{Uncovering the List Structure}
We've gotten rid of higher order functions --- in doing so, we've turned a deeply nested function (the continuation) into a deeply nested constructor that might look like
\begin{minted}[bgcolor=backcolour, linenos]{ocaml}
(K1(n1, K2(n2, ID)))
\end{minted}
We don't seem to have made much progress, but notice that the nesting now clearly has a list structure! We can treat \texttt{ID} as the empty list, and \emph{cons} of \texttt{K2} and \texttt{K1} as appropriate. That is, 
\begin{minted}[bgcolor=backcolour, linenos]{ocaml}
K1(n1) :: K2(n2) :: []
\end{minted}

While we're at it, let's give \texttt{K1} and \texttt{K2} more meaningful names. Let's call \texttt{K1} ``\texttt{SUB2}'' and \texttt{K2} ``\texttt{PLUS}''. 

\begin{code}[Uncovering the list structure]
\label{code:defun-list-fib}
\begin{minted}[bgcolor=backcolour, linenos]{ocaml}
type tag = SUB2 of int | PLUS of int
type tag_list_cont = taglist

let rec apply_tag_list_cont k v = match k, v with
  | [], v -> v
  | SUB2(n)::k, x -> fib_cps_defun (n-2) ((PLUS x)::k))
  | PLUS(x)::k, y -> apply_cont k (x+y)

and fib_cps_defun_tags n k = 
  match n with
    | 0 -> apply_tag_list_cont k 1
    | 1 -> apply_tag_list_cont k 1
    | n -> fib_cps_defun_tags n-1 ((SUB2 n)::k))

let fib_3 n = fib_cps_defun_tags n []
\end{minted}
\end{code}

And so we have our call stack, as described at the start of this chapter.

\subsection{Benefits of Defunctionalisation}\label{subsection:defun-benefits}
In this section, we'll consider the benefits of defunctionalisation beyond our very narrow use case.

\subsubsection{No higher-order functions}
Getting rid of higher order functions is useful because while higher order functions make code easier to \emph{write}, they make code harder to \emph{read} --- both for humans, and for machines. This is the main advantage: all other advantages are \emph{consequences} of this.

\subsubsection{Every function is named}
This is similar to how CPS transform forces you to name every intermediate result. 

\subsubsection{All values are data}
For example, one consequence is that all values are data --- since higher-order functions are the only way to get functions as values. This is \emph{extremely} useful for analysis, and thus optimisation. There are many uses, but one important one is that \emph{value equality} becomes a lot more straightforward. 

As you will have learnt in {\sffamily Part IB Semantics of Programming Languages}, the problem with functions when it comes to equality is that function definitions are \emph{intensional} rather than \emph{extensional}. Basically, it means that the functions \texttt{fun x -> x + x} and \texttt{fun x -> 2 * x} are different, even though their \emph{graphs} are the same. This complication goes away when all values are data.

\subsubsection{All control flow is first order}
Another advantage (a consequence of the fact that there are no higher-order functions) is that all control flow is first order (i.e. there is no \emph{indirection}). This makes \emph{analysis} and therefore \emph{optimisation} significantly easier. For example, \emph{reachability analysis} can be used for unreachable procedure elimination. That is, in the following code, the function \texttt{h} is never called, and can therefore be removed to make the program smaller.

\begin{code}[An example of unreachable procedure elimination in \texttt{C}]
\label{code:unreachable-c}
\begin{minted}[bgcolor=backcolour, linenos]{C}
int main () {
  return f (1)
}

int f(int x) {
  return g(x) + 1
}

int g(int x) {
  return -x
}

int h(int x) {  // Unreachable
  return x
}
\end{minted}
\end{code}

The task of unreachable procedure elimination becomes a lot harder when control flow is not first order. For example,

\begin{code}[Unreachable procedure elimination with function indirection in \texttt{C}]
\label{code:unreachable-indirection-c}
\begin{minted}[bgcolor=backcolour, linenos]{C}
int main () {
  return f(1, &g)
}

int f(int x, (*lambda)(int, int)) {
  return lambda(x) + 1
}

int g(int x) {
  return -x
}

int h(int x) {  // Still unreachable
  return x
}
\end{minted}
\end{code}

\texttt{h} is still unreachable, but figuring this out is now no longer as easy. Indirection makes analysis harder. This is a major theme in {\sffamily Part II Optimising Compilers}.

\section{Clean Up}\label{section:clean-up}
We have our call stack, but we should do a couple of clean up steps before calling it a day. In \Cref{subsection:mutrec} we'll show how to eliminate mutual recursion, and in \Cref{subsection:refactoring} we'll perform a very simple refactoring to more cleanly separate concerns across functions.

\subsection{Mutual Recursion}\label{subsection:mutrec}
One feature that we'd like to get rid of is \emph{mutual recursion}. \texttt{apply\_tag\_list\_cont} will call \texttt{fib\_cps\_defun\_tags} and vice versa, so the two functions \textbf{have} to be defined together. Eliminating mutual recursion is \emph{easy}, both conceptually and in practice. Let's see it in a simple setting first. Here are two mutually recursive functions for determining the parity of a number.

\begin{code}[A set of mutually recursive functions]
\label{code:mutual-recursion}
\begin{minted}[bgcolor=backcolour, linenos]{ocaml}
let rec is_even n = n=0 || is_odd(n-1)
and is_odd n = n <> 0 && is_even(n-1)
\end{minted}
\end{code}

We'll convert this mutual recursion into single recursion by defining a single function, \texttt{is}, and a constructor, \texttt{eo}.

\begin{code}[Eliminating mutual recursion]
\label{code:mutual-recursion}
\begin{minted}[bgcolor=backcolour, linenos]{ocaml}
type eo = Even | Odd
let is f n = 
  match f with
    | Even -> n=0  || is Odd  (n-1)
    | Odd  -> n<>0 && is Even (n-1)
\end{minted}
\end{code}

The abstract idea that we've shown in this concrete example is that we replace the \emph{two} functions with a single function (in this case replacing \texttt{is\_even} and \texttt{is\_odd} with \texttt{is}), and use a ``tag'' or ``index'' (in this case \texttt{f}) to parameterise the function behaviour.

Applying this technique to our fibonacci function, we get

\begin{code}[Eliminating mutual recursion for our fibonacci function]
\label{code:defun-list-fib}
\begin{minted}[bgcolor=backcolour, linenos]{ocaml}
type state_type = FIB (*means we should act like fib_cps_defun_tags*)
                | APP (*means we should act like apply_tag_list_cont*)

let rec eval = function
  | FIB, 0, k -> eval (APP, 1, k)
  | FIB, 1, k -> eval (APP, 1, k)
  | FIB, n, k -> eval (FIB, n-1, SUB2 n :: k)
  | APP, x, SUB2 n :: k -> eval (FIB, n-2, PLUS x :: k)
  | APP, y, PLUS x :: k -> eval (APP, x + y, k)
  | APP, v, [] -> v

let fib_4 n = eval (FIB, n, [])
\end{minted}
\end{code}


\subsection{Refactoring}\label{subsection:refactoring}
Finally, \texttt{eval} is slightly clunky because it's a big function that calls itself recursively, but also has a lot of cases. Ideally, we want a function to either be a large pattern match or call itself recursively. We can do this by using a simple refactoring. 

\begin{code}[Refactoring our fibonacci function]
\label{code:defun-list-fib}
\begin{minted}[bgcolor=backcolour, linenos]{ocaml}
let step = function
  | FIB, 0, k -> APP, 1, k
  | FIB, 1, k -> APP, 1, k
  | FIB, n, k -> FIB, n-1, SUB2 n :: k
  | APP, x, SUB2 n :: k -> FIB, n-2, PLUS x :: k
  | APP, y, PLUS x :: k -> APP, x + y, k
  | _ -> fail_with "step: runtime error!"

let rec driver = function
  | APP, v, [] -> v
  | state -> driver (step state)

let fib_5 n = driver (FIB, n, [])
\end{minted}
\end{code}


\section{Recap}
Let's end the chapter by considering where we started, and where we ended. We started with a bog-standard fibonacci function, applied CPS transformation and defunctionalisation, and then eliminated mutual recursion and performed some refactoring. In doing so, we transformed plain, bog-standard \texttt{OCaml} code into something where an explicit list acts like the call stacks we described in \Cref{section:call-stacks}.