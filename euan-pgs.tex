% Title Setup
%%%%%%%%%%%%%%%%%%%%%%%%%%%%%%
\addtokomafont{subtitle}{\Large}
\setkomafont{author}{\Large\scshape}
\setkomafont{date}{\Large\normalsize}


% Theorem Setup
%%%%%%%%%%%%%%%%%%%%%%%%%%%%%%

% Colours
\definecolor{mdgrey}{HTML}{8C8C8E}
\definecolor{examplebg}{HTML}{FFFAF6}
\definecolor{examplefg}{HTML}{8A3D45}
\definecolor{notefg}{HTML}{303034}
\definecolor{terminologybg}{HTML}{FFFAF6}
\definecolor{terminologyfg}{HTML}{58355E}
\definecolor{definitionbg}{HTML}{FFFAF6}
\definecolor{definitionfg}{HTML}{274F6A}
\definecolor{theorembg}{HTML}{FFFAF6}
\definecolor{theoremfg}{HTML}{874569}
\definecolor{theorembody}{HTML}{5E5F63}
\definecolor{proofbg}{HTML}{F2F2F2}
\definecolor{prooffg}{HTML}{808080}
\definecolor{markerbg}{HTML}{E2204C}
\definecolor{markerfg}{HTML}{FFFFFF}
\definecolor{packagebg}{HTML}{F5F5F8}
\definecolor{packagefg}{HTML}{1C6767}
\definecolor{packagebd}{HTML}{27282C}
\definecolor{linkcolor}{HTML}{019EC9}
\definecolor{highlight}{HTML}{8A0630}
\definecolor{darkred}{HTML}{000000}
\definecolor{darkest}{HTML}{000000}
\definecolor{codegrey}{rgb}{0.6,0.6,0.6}
\definecolor{codemagenta}{HTML}{ae2012}
\definecolor{codeteal}{HTML}{ca6702}
\definecolor{codepurple}{HTML}{005f73}
\definecolor{backcolour}{HTML}{F8F8FB}
\definecolor{background}{HTML}{F0F0F4}
\definecolor{darkest}{gray}{0.3}
\definecolor{nonterminal}{HTML}{006B71}
\definecolor{selected}{HTML}{00AB97}
\definecolor{grey}{gray}{0.5}
\definecolor{lightgrey}{gray}{0.9}
\definecolor{verylightgrey}{rgb}{0.96,0.96,0.965}

\mdfsetup{
    skipabove=0.8\baselineskip,
    skipbelow=0.3\baselineskip,
    innertopmargin=10pt,
    innerbottommargin=14pt,
    innerleftmargin=12pt,
    innerrightmargin=12pt,
    hidealllines=true
}

% Proof
\mdfdefinestyle{mdproofbox}{
    skipabove=0.4\baselineskip,
    skipbelow=0.3\baselineskip,
    backgroundcolor=proofbg,
    linecolor=prooffg,
    linewidth=1.5pt,
    topline=false,
    innerbottommargin=6pt,
    rightline=false,
    bottomline=false,
    leftline=true
}

% Example
\declaretheoremstyle[
    headfont=\ttfamily\bfseries\scriptsize\color{mdgrey},
    notefont=\normalsize\sffamily\bfseries\color{examplefg},
    notebraces={}{},
    headformat={{\NAME \ \NUMBER} \\[3pt] \NOTE},
    mdframed={
        backgroundcolor=examplebg,
        nobreak=false,
        roundcorner=2pt
    },
    headpunct={\\[9pt]},
    postheadspace={3pt}
]{thmexamplebox}

% Definition
\declaretheoremstyle[
    headfont=\ttfamily\bfseries\scriptsize\color{mdgrey},
    notefont=\normalsize\sffamily\bfseries\color{definitionfg},
    notebraces={}{},
    headformat={{\NAME \ \NUMBER} \\[3pt] \NOTE},
    mdframed={
        backgroundcolor=definitionbg,
        nobreak=false,
        roundcorner=2pt
    },
    % headpunct={\\[9pt]},
    postheadspace={3pt}
]{thmdefinitionbox}

% Terminology
\declaretheoremstyle[
    headfont=\ttfamily\bfseries\scriptsize\color{mdgrey},
    notefont=\normalsize\sffamily\bfseries\color{terminologyfg},
    notebraces={}{},
    headformat={{\NAME \ \NUMBER} \\[3pt] \NOTE},
    mdframed={
        backgroundcolor=definitionbg,
        nobreak=false,
        roundcorner=2pt
    },
    headpunct={\\[9pt]},
    postheadspace={3pt}
]{thmterminologybox}

% Note
\declaretheoremstyle[
    headfont=\ttfamily\bfseries\scriptsize\color{mdgrey},
    notefont=\normalsize\sffamily\bfseries\color{notefg},
    notebraces={}{},
    headformat={{\NAME \ \NUMBER} \\[3pt] \NOTE},
    mdframed={
        backgroundcolor=definitionbg,
        nobreak=false,
        roundcorner=2pt
    },
    headpunct={\\[9pt]},
    postheadspace={3pt}
]{thmnotebox}

% Package
\declaretheoremstyle[
    headfont=\large\sffamily\bfseries\color{packagefg},
    notebraces={}{},
    headformat={\vspace{0mm}\\ {\NAME}},
    mdframed={
        backgroundcolor=packagebg,
        nobreak=false,
        roundcorner=2pt
    },
    bodyfont=\color{packagebd},
    headpunct={\\[9pt]},
    postheadspace={3pt}
]{thmpackagebox}

% Package
\declaretheoremstyle[
    headfont=\Large\sffamily\bfseries\color{packagebg},
    notebraces={}{},
    headformat={{\NAME}},
    mdframed={
        backgroundcolor=packagefg,
        nobreak=false,
        roundcorner=2pt
    },
    bodyfont=\color{white!80},
    headpunct={\\[9pt]},
    postheadspace={3pt}
]{thmpackageselectedbox}

% Note
\declaretheoremstyle[
    headfont=\sffamily\bfseries\color{markerfg},
    notebraces={}{},
    headformat={{\NAME}},
    mdframed={
        backgroundcolor=markerbg,
        nobreak=false,
        roundcorner=2pt
    },
    bodyfont=\color{markerfg},
    headpunct={\\[9pt]},
    postheadspace={3pt}
]{thmmarkerbox}

% Theorem
\declaretheoremstyle[
    headfont=\ttfamily\bfseries\scriptsize\color{mdgrey},
    notefont=\normalsize\sffamily\bfseries\color{theoremfg},
    notebraces={}{},
    headformat={{\NAME \ \NUMBER} \\[3pt] \NOTE},
    mdframed={
        backgroundcolor=definitionbg,
        nobreak=false,
        roundcorner=2pt
    },
    headpunct={\\[9pt]},
    postheadspace={3pt}
]{thmtheorembox}


\newcommand{\thistheoremname}{}

\ifeuanfancytheorems
    \declaretheorem[style=thmtheorembox,name=Theorem,numberwithin=section]{theorem}
    \declaretheorem[style=thmtheorembox,name=Lemma,sibling=theorem]{lemma}
    \declaretheorem[style=thmtheorembox,name=Fact,sibling=theorem]{fact}
    \declaretheorem[style=thmtheorembox,name=Proposition,sibling=theorem]{proposition}
    \declaretheorem[style=thmtheorembox,name=Corollary,sibling=theorem]{corollary}
    \declaretheorem[style=thmexamplebox,name=Example,sibling=theorem]{example}
    \declaretheorem[style=thmdefinitionbox,name=Definition,sibling=theorem]{definition}
    \declaretheorem[style=thmdefinitionbox,name=Axiom,sibling=theorem]{axiom}
    \declaretheorem[style=thmtheorembox,name=Theorem,numbered=yes]{theorem*}
    \declaretheorem[style=thmtheorembox,name=Lemma,numbered=no]{lemma*}
    \declaretheorem[style=thmtheorembox,name=Claim,numbered=no]{claim*}
    \declaretheorem[style=thmtheorembox,name=Fact,numbered=no]{fact*}
    \declaretheorem[style=thmtheorembox,name=Proposition,numbered=no]{proposition*}
    \declaretheorem[style=thmtheorembox,name=Corollary,numbered=no]{corollary*}
    \declaretheorem[style=thmexamplebox,name=Example,numbered=no]{example*}
    \declaretheorem[style=thmexamplebox,name=Question,numbered=no]{question}
    \declaretheorem[style=thmexamplebox,name=Problem,numberwithin=section]{problem}
    \declaretheorem[style=thmexamplebox,name=Problem,numbered=no]{problem*}
    \declaretheorem[style=thmdefinitionbox,name=Definition,numbered=yes]{definition*}
    \declaretheorem[style=thmdefinitionbox,name=Axiom,numbered=no]{axiom*}
    \declaretheorem[style=thmnotebox,name=Note,numbered=no]{note*}
    \declaretheorem[style=thmterminologybox, name=Terminology, numbered=no]{terminology*}

    \declaretheorem[style=thmtheorembox,name=\protect\thistheoremname,numbered=yes]
    {theorembox_style*}
    \declaretheorem[style=thmexamplebox,name=\protect\thistheoremname,numbered=no]{examplebox_style*}
    \declaretheorem[style=thmdefinitionbox,name=Definition,numberwithin=section]{definitionbox_style*}
    \declaretheorem[style=thmnotebox, name=Note, numberwithin=section]{notebox_style*}
    \declaretheorem[style=thmmarkerbox, name=\protect\thistheoremname]{markerbox_style*}

    \declaretheorem[style=thmpackagebox, name=\protect\thistheoremname]{packagebox_style*}


\else
    \newtheorem{theorem}{Theorem}[section]
    \newtheorem{lemma}[theorem]{Lemma}
    \newtheorem{fact}[theorem]{Fact}
    \newtheorem{corollary}[theorem]{Corollary}
    \newtheorem{proposition}[theorem]{Proposition}
    \newtheorem*{theorem*}{Theorem}
    \newtheorem*{lemma*}{Lemma}
    \newtheorem*{fact*}{Fact}
    \newtheorem*{corollary*}{Corollary}
    \newtheorem*{proposition*}{Proposition}

    \theoremstyle{definition}

    \newtheorem{example}[theorem]{Example}
    \newtheorem{definition}[theorem]{Definition}
    \newtheorem{axiom}[theorem]{Axiom}
    \newtheorem*{example*}{Example}
    \newtheorem*{definition*}{Definition}
    \newtheorem*{axiom*}{Axiom}
    \newtheorem{terminology*}{Terminology}

    \newtheorem{problem}[theorem]{Problem}
    \newtheorem*{problem*}{Problem}

    \newtheorem*{theorembox_style*}{\protect\thistheoremname}
    \newtheorem*{examplebox_style*}{\protect\thistheoremname}
    \newtheorem*{definitionbox_style*}{\protect\thistheoremname}
\fi

\newenvironment{theorembox*}[1]
    {\renewcommand{\thistheoremname}{#1}
    \begin{theorembox_style*}}
    {\end{theorembox_style*}}

\newenvironment{examplebox*}[1]
    {\renewcommand{\thistheoremname}{#1}
    \begin{examplebox_style*}}
    {\end{examplebox_style*}}
   
\newenvironment{definitionbox*}[1]
    {\renewcommand{\thistheoremname}{#1}
    \begin{definitionbox_style*}[#1]}
    {\end{definitionbox_style*}}

\newenvironment{notebox*}[1]
    {\renewcommand{\thistheoremname}{#1}
    \begin{notebox_style*}[#1]}
    {\end{notebox_style*}}

\newenvironment{markerbox*}[1]
    {\renewcommand{\thistheoremname}{#1}
    \begin{markerbox_style*}[#1]}
    {\end{markerbox_style*}}

\newenvironment{packagebox*}[1]
    {\renewcommand{\thistheoremname}{#1}
    \begin{packagebox_style*}[#1]}
    {\end{packagebox_style*}}

\ifeuanfancyproofs
\surroundwithmdframed[style=mdproofbox]{proof}
\else
\AtBeginEnvironment{proof}{\vspace{-0.5\baselineskip}}
\fi

\theoremstyle{definition}

\newtheorem{conjecture}[theorem]{Conjecture}
\newtheorem{answer}[theorem]{Answer}
\newtheorem{case}[theorem]{Case}
% \newtheorem{question}[theorem]{Question}
\newtheorem{exercise}[theorem]{Exercise}
% \newtheorem{problem}[theorem]{Problem}
\newtheorem*{conjecture*}{Conjecture}
\newtheorem*{answer*}{Answer}
\newtheorem*{case*}{Case}
\newtheorem*{joke*}{Joke}
\newtheorem*{question*}{Question}
\newtheorem*{exercise*}{Exercise}
\newtheorem*{notation}{Notation}
\newtheorem*{remark}{Remark}


% Header Setup
%%%%%%%%%%%%%%%%%%%%%%%%%%%%%%

\usepackage[headsepline]{scrlayer-scrpage}
\renewcommand{\headfont}{}
\addtolength{\textheight}{3.14cm}
\setlength{\footskip}{0.5in}
\setlength{\headsep}{10pt}
\ihead{\footnotesize\textbf{\@author} (\@date)}
\automark{section}
\chead{}
\ohead{\footnotesize\textbf{\@title}}
\cfoot{\pagemark}

% Fancy Section and Chapter Heads 
%%%%%%%%%%%%%%%%%%%%%%%%%%%%%%%%%%%

%\usepackage[tocindentauto]{tocbasic}
%\usetocstyle{KOMAlike}

\newenvironment{aside}[1]{
    \rule{\textwidth}{0.025cm}
    \vspace{-1.75\baselineskip}
    \subsection*{#1}}
{\rule{\textwidth}{0.025cm}}