\begin{tikzpicture}[>=stealth',shorten >=1pt,auto,node distance=3cm]
%% s1: 1, 2, 3, 5, 8
%% s2: 2, 3, 4, 5, 7, 8, 9
%% s3: 2, 3, 5, 6, 7, 8
%% s4: 2, 3, 5, 6, 7, 8, 10
%% s5: 2, 3, 5, 6, 7, 8, 11
\node[state] (s1) { \parbox{6ex}{\centering\footnotesize\itshape 1,2,3,\\5,8} };
\node[state] (s2) [above right of=s1] { \parbox{6ex}{\centering\footnotesize\itshape 2,3,4,5,\\7,8,9} };
\node[state] (s3) [below right of=s1] { \parbox{6ex}{\centering\footnotesize\itshape 2,3,5,6,\\7,8}};
\node[state] (s4) [right of=s3] { \parbox{6ex}{\centering\footnotesize\itshape 2,3,5,6,\\7,8,10}};
\node[state,accepting] (s5) [above of=s4] { \parbox{6ex}{\centering\footnotesize\itshape 2,3,5,6,\\7,8,11}};

%% s1 -a-> s2
\path[->] (s1) edge node {a} (s2);
%% s1 -b-> s3
\path[->] (s1) edge node {b} (s3);
%% s2 -a-> s2
\path[->] (s2) edge[loop right] node {a} (s2);
%% s3 -a-> s2
\path[->] (s3) edge node {a} (s2);
%% s2 -b-> s4
\path[->] (s2) edge[bend right=20] node[below] {b} (s4);
%% s3 -b-> s3
\path[->] (s3) edge[loop right] node {b} (s3);
%% s4 -a-> s2
\path[->] (s4) edge[bend right=15] node[above] {a} (s2);
%% s4 -b-> s5
\path[->] (s4) edge node {b} (s5);
%% s5 -a-> s2
\path[->] (s5) edge node {a} (s2);
%% s5 -b-> s3
\path[->] (s5) edge node {b} (s3);

%% \path[->] (after-a) edge[bend right=12] node[right] { b }  (after-b);
%% \path[->] (after-b) edge[bend right=12] node[right] { a }  (after-a);
%% \path[->] (after-a) edge[loop right] node { a }  (after-a) ; 
%% \path[->] (after-b) edge[loop right] node { b }  (after-b) ; 
\end{tikzpicture}